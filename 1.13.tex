%!TEX root = main.tex
Los ejercicios de esta sección son más bien opcionales ya que Korgi no suele evaluar demostraciones en este curso, sin embargo para las personas que quizá se les dificulte demostrar estas afirmaciones y esté interesado en aprender se hace la solución de los mismos.\\

\textbf{Punto 1:}

\begin{itemize}
    \item[2)]
    \begin{proof}
        Usando la definición de unión y de reflexión de un lenguaje tenemos que:

\begin{align*}
    (A \cup B)^R&=\{u^R: u \in A \text{ o } u\in B\}\\
    &=\{x^R: x\in A\}\cup\{y^R: y \in B\}\\
    &=A^R\cup B^R
\end{align*}  
    \end{proof}

    \item[3)] 

    \begin{proof}
        En este caso es similar el argumento solo que cambiamos la ''o'' por una ''y'' ya que es una intersección y por tanto los elementos deben estar en ambos conjuntos:

        \begin{align*}
              (A \cap B)^R&=\{u^R: u \in A \text{ y } u\in B\}\\
    &=\{x^R: x\in A\}\cap\{y^R: y \in B\}\\
    &=A^R\cap B^R
        \end{align*}
    \end{proof}

    \item[4)]\begin{proof}
        Usando la definición de reflexión de un lenguaje tenemos que:

        $$(A^R)^R=\{(u^R)^R:u\in A\}$$

        Es decir es la reflexión de la reflexión de todas las cadenas de A, y nosotros ya sabemos que la reflexión de la reflexión de una cadena es la misma cadena luego:

        $$(A^R)^R=\{u: u \in A\}=A$$

    \end{proof}

     \item[6)] 
   \begin{proof}En este caso seguiremos el mismo modelo de prueba que se usa en las notas de clase para la propiedad 5
        $$
    \begin{aligned}
    x \in\left(A^+\right)^R & \Longleftrightarrow x=u^R, \text { donde } u \in A^+ \\
    & \Longleftrightarrow x=\left(u_1 \cdot u_2 \cdots u_n\right)^R, \text { donde los } u_i \in A, n \geq 1 \\
    & \Longleftrightarrow x=u_n^R \cdot u_2^R \cdots u_1^R, \text { donde } \operatorname{los} u_i \in A, n \geq 1 \\
    & \Longleftrightarrow x \in\left(A^R\right)^+ .
    \end{aligned}
    $$
   \end{proof}
\end{itemize}

\textbf{Punto 2:} CLARAmente se pueden generalizar las propiedades 2 y 3 ya que la unión y la intersección se comportan bien 2 a 2, es decir si tenemos que calcular la reflexión de la unión de $n$ conjuntos, podemos hacerlo con los primeros dos y luego con los siguientes dos y así hasta acabar o que nos quede solo uno y pues en ese caso calculamos la reflexión y unimos todo o en el otro caso intersectamos.\\

\begin{align*}
\displaystyle\left(\bigcup_{i \geq 0} A_i\right)^R&=(A_1\cup A_2)^R\cup (A_3\cup A_4\cup A_5\cup ...)^R\\
&=A_1^R\cup A_2^R\cup (A_3\cup A_4)^R\cup (A_5\cup A_6\cup A_7\cup ...)^R
\end{align*}

Y continuando así llegamos a:

$$\displaystyle\left(\bigcup_{i \geq 0} A_i\right)^R=A_1^R\cup A_2^R\cup A_3^R\cup...\cup A_n^R...$$

De manera análoga se ve la intersección.

\hfill $\blacklozenge$