%!TEX root = ../main.tex

Previamente habíamos mencionado que a partir de un AFD con una condición podíamos construir un AFD para la negación de esa condición cambiando los estados de aceptación. En esta sección eso sera lo que haremos.\\ 

\textbf{Punto 1: }El lenguaje de todas las cadenas que no contienen la subcadena $bc.$ Alfabeto: $\{a,b,c\}$. \\

Construyamos primero un AFD tal que acepte todas las cadenas que contienen la subcadena $bc$. Para esto forzamos $bc$ como aceptación y consideramos los bucles necesarios.
\begin{basedtikz}
\centering
    \begin{tikzpicture}[node distance = 2.5cm, on grid, auto]
        \node (q0) [state, initial] {$q_0$};
        \node (q1) [state, right of=q0] {$q_1$};
        \node (q2) [state, right of=q1, accepting] {$q_2$};
        \path[thick]
        (q0) edge [loop above] node [above] {$a,c$} ()
        (q1) edge [loop above] node [above] {$b$} ()
        (q2) edge [loop above] node [above] {$a,b,c$} ()
        (q0) edge [bend left] node [above] {$b$} (q1)
        (q1) edge [bend left] node [below] {$a$} (q0)
        (q1) edge node [above] {$c$} (q2);
    \end{tikzpicture}
\end{basedtikz}
Ahora simplemente cambiamos los estados de aceptación, es decir $q_0$ y $q_1$ se vuelven de aceptación mientras que $q_2$ deja de serlo. De esta manera obtenemos el AFD que acepta el lenguaje deseado.
\begin{basedtikz}
\centering
    \begin{tikzpicture}[node distance = 2.5cm, on grid, auto]
        \node (q0) [state, initial, accepting] {$q_0$};
        \node (q1) [state, right of=q0, accepting] {$q_1$};
        \node (q2) [state, right of=q1] {$q_2$};
        \path[thick]
        (q0) edge [loop above] node [above] {$a,c$} ()
        (q1) edge [loop above] node [above] {$b$} ()
        (q2) edge [loop above] node [above] {$a,b,c$} ()
        (q0) edge [bend left] node [above] {$b$} (q1)
        (q1) edge [bend left] node [below] {$a$} (q0)
        (q1) edge node [above] {$c$} (q2);
    \end{tikzpicture}
\end{basedtikz}

\textbf{Punto 2: }El lenguaje de todas las cadenas que no tienen tres unos consecutivos. Alfabeto: $\{0,1\}$. \\

Primero construyamos el AFD que acepte todas las cadenas con tres unos consecutivos.
\begin{basedtikz}
\centering
    \begin{tikzpicture}[node distance = 2.5cm, on grid, auto]
        \node (q0) [state, initial] {$q_0$};
        \node (q1) [state, right of=q0] {$q_1$};
        \node (q2) [state, right of=q1] {$q_2$};
        \node (q3) [state, right of=q2, accepting] {$q_3$};
        \path[thick]
        (q0) edge [loop above] node [above] {$0$} ()
        (q0) edge [bend left] node [above] {$1$} (q1)
        (q1) edge [bend left] node [below] {$0$} (q0)
        (q1) edge node [above] {$1$} (q2)
        (q2) edge node [above] {$1$} (q3)
        (q2) edge [bend left=60] node [below] {$0$} (q0)
        (q3) edge [loop above] node [above] {$0,1$} ();
    \end{tikzpicture}
\end{basedtikz}
Luego el AFD que acepta el lenguaje deseado es:
 
\begin{basedtikz}
\centering
    \begin{tikzpicture}[node distance = 2.5cm, on grid, auto]
        \node (q0) [state, initial, accepting] {$q_0$};
        \node (q1) [state, right of=q0, accepting] {$q_1$};
        \node (q2) [state, right of=q1, accepting] {$q_2$};
        \node (q3) [state, right of=q2] {$q_3$};
        \path[thick]
        (q0) edge [loop above] node [above] {$0$} ()
        (q0) edge [bend left] node [above] {$1$} (q1)
        (q1) edge [bend left] node [below] {$0$} (q0)
        (q1) edge node [above] {$1$} (q2)
        (q2) edge node [above] {$1$} (q3)
        (q2) edge [bend left=60] node [below] {$0$} (q0)
        (q3) edge [loop above] node [above] {$0,1$} ();
    \end{tikzpicture}
\end{basedtikz}


\textbf{Punto 3: }El lenguaje de todas las cadenas que no terminan en 01. Alfabeto: $\{0,1\}$. \\

Primero construyamos el AFD que acepta todas las cadenas que terminan en 01.
\begin{basedtikz}
\centering
    \begin{tikzpicture}[node distance = 2.5cm, on grid, auto]
        \node (q0) [state, initial] {$q_0$};
        \node (q1) [state, right of=q0] {$q_1$};
        \node (q2) [state, right of=q1, accepting] {$q_2$};
        \path[thick]
        (q0) edge [loop above] node [above] {$1$} ()
        (q1) edge [loop above] node [above] {$0$} ()
        (q1) edge [bend left] node [above] {$1$} (q2)
        (q2) edge [bend left] node [below] {$0$} (q1)
        (q0) edge node [above] {$0$} (q1)
        (q2) edge [bend left=60] node [below] {$1$} (q0);
    \end{tikzpicture}
\end{basedtikz}
Luego el AFD que acepta el lenguaje deseado es:
\begin{basedtikz}
\centering
    \begin{tikzpicture}[node distance = 2.5cm, on grid, auto]
        \node (q0) [state, initial, accepting] {$q_0$};
        \node (q1) [state, right of=q0, accepting] {$q_1$};
        \node (q2) [state, right of=q1] {$q_2$};
        \path[thick]
        (q0) edge [loop above] node [above] {$1$} ()
        (q1) edge [loop above] node [above] {$0$} ()
        (q1) edge [bend left] node [above] {$1$} (q2)
        (q2) edge [bend left] node [below] {$0$} (q1)
        (q0) edge node [above] {$0$} (q1)
        (q2) edge [bend left=60] node [below] {$1$} (q0);
    \end{tikzpicture}
\end{basedtikz}

\textbf{Punto 4: }El lenguaje de todas las cadenas que no terminan en 22. Alfabeto: $\{0,1,2\}$. \\

Primero construyamos el AFD que acepta todas las cadenas que terminan en 22.
 
\begin{basedtikz}
\centering
    \begin{tikzpicture}[node distance = 2.5cm, on grid, auto]
        \node (q0) [state, initial] {$q_0$};
        \node (q1) [state, right of=q0] {$q_1$};
        \node (q2) [state, right of=q1, accepting] {$q_2$};
        \path[thick]
        (q0) edge [loop above] node [above] {$0,1$} ()
        (q2) edge [loop above] node [above] {$2$} ()
        (q0) edge [bend left] node [above] {$2$} (q1)
        (q1) edge [bend left] node [below] {$0,1$} (q0)
        (q1) edge node [above] {$2$} (q2)
        (q2) edge [bend left=60] node [below] {$0,1$} (q0);
    \end{tikzpicture}
\end{basedtikz}
Luego el AFD que acepta el lenguaje deseado es:
\begin{basedtikz}
\centering
    \begin{tikzpicture}[node distance = 2.5cm, on grid, auto]
        \node (q0) [state, initial, accepting] {$q_0$};
        \node (q1) [state, right of=q0, accepting] {$q_1$};
        \node (q2) [state, right of=q1] {$q_2$};
        \path[thick]
        (q0) edge [loop above] node [above] {$0,1$} ()
        (q2) edge [loop above] node [above] {$2$} ()
        (q0) edge [bend left] node [above] {$2$} (q1)
        (q1) edge [bend left] node [below] {$0,1$} (q0)
        (q1) edge node [above] {$2$} (q2)
        (q2) edge [bend left=60] node [below] {$0,1$} (q0);
    \end{tikzpicture}
\end{basedtikz}

Note que en los puntos 1 y 2, los estados $q_2$ y $q_3$ se volvieron limbo respectivamente y es posible quitarlos para una presentación simplificada del AFD, pero es preferible no retirarlos con el propósito de observar correctamente el proceso del complemento.

$\hfill \blacklozenge$ 
