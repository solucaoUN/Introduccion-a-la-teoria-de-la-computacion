%!TEX root = ../main.tex

Si bien previamente habíamos observado como los AFN eran bastante fáciles de construir, en esta sección nos daremos cuenta de que existe incluso construcciones aun mas sencillas, que son los AFN-$\lambda$. En esta sección veremos como construirlos y luego observaremos su relevancia.\\

\textbf{Punto 1: }Sea $\Sigma=\{a,b\}.$ Diseñar AFN-$\lambda$ que acepten los siguientes lenguajes:

\begin{itemize}[label={$\bullet$}]
    \item $(ab\cup b)^*ab^*a^*.$\\
     
     Para este autómata realizaremos una construcción en serie, concatenando los autómatas respectivos de cada parte del lenguaje, note que $a$ es la mínima cadena de aceptación.
    \begin{basedtikz}
    \centering
       \begin{tikzpicture}[node distance = 2.5cm, on grid, auto]
       \node (q0) [state, initial] {$q_0$};
       \node (q1) [state, below of=q0] {$q_1$};
       \node (q2) [state, right of=q0] {$q_2$};
       \node (q3) [state, right of=q2] {$q_3$};
       \node (q4) [state, right of=q3, accepting] {$q_4$};
       \path[thick]
       (q0) edge [loop above] node [above] {$b$} ()
       (q0) edge [bend right] node [left] {$a$} (q1)
       (q1) edge [bend right] node [right] {$b$} (q0)
       (q0) edge node [above] {$\lambda$} (q2)
       (q2) edge node [above] {$a$} (q3)
       (q3) edge [loop above] node [above] {$b$} ()
       (q3) edge node [above] {$\lambda$} (q4)
       (q4) edge [loop above] node [above] {$a$} ();
       \end{tikzpicture} 
    \end{basedtikz}

    \item $ab^*\cup ba^*\cup b(ab\cup ba)^*.$\\
     
     Basta con realizar una construcción en paralelo para cada expresión de la unión. Es decir construimos un AFD para $ab^*,ba^*$ y $b(ab\cup ba)^*$ respectivamente y luego realizamos la construcción en paralelo.
    \begin{basedtikz}
    \centering
       \begin{tikzpicture}[node distance = 2.5cm, on grid, auto]
       \node (q0) [state, initial] {$q_0$};
       \node (q1) [state, right of=q0] {$q_1$};
       \node (q2) [state, right of=q1, accepting] {$q_2$};
       \node (q3) [state, above of=q1] {$q_3$};
       \node (q4) [state, right of=q3, accepting] {$q_4$};
       \node (q5) [state, below of=q1] {$q_5$};
       \node (q6) [state, right of=q5, accepting] {$q_6$};
       \node (q7) [state, right of=q6] {$q_7$};
       \node (q8) [state, below of=q6] {$q_6$};
       \path[thick]
       (q0) edge node [above] {$\lambda$} (q1)
       (q0) edge node [above] {$\lambda$} (q3)
       (q0) edge node [below] {$\lambda$} (q5)
       (q1) edge node [above] {$b$} (q2)
       (q2) edge [loop right] node [right] {$a$} ()
       (q3) edge node [above] {$a$} (q4)
       (q4) edge [loop right] node [right] {$b$} ()
       (q5) edge node [above] {$b$} (q6)
       (q6) edge [bend left] node [above] {$a$} (q7)
       (q7) edge [bend left] node [below] {$b$} (q6)
       (q6) edge [bend right] node [left] {$b$} (q8)
       (q8) edge [bend right] node [right] {$a$} (q6);
       \end{tikzpicture} 
    \end{basedtikz}

    \item $(a\cup aba)^*b^*(ab\cup ba)^*a^*.$ \\

    Nuevamente realizaremos una construcción en serie, note que la cadena $\lambda$ es la mínima que se acepta.
    \begin{basedtikz}
    \centering
        \begin{tikzpicture}[node distance = 2.5cm, on grid, auto]
         \node (q0) [state, initial] {$q_0$};
         \node (q1) [state, below of=q0] {$q_1$};
         \node (q2) [state, right of=q1] {$q_2$};
         \node (q3) [state, right of=q0] {$q_3$};
         \node (q4) [state, right of=q3] {$q_4$};
         \node (q5) [state, above of=q4] {$q_5$};
         \node (q6) [state, below of=q4] {$q_6$};
         \node (q7) [state, right of=q4, accepting] {$q_7$};
         \path[thick]
         (q0) edge [loop above] node [above] {$a$} ()
         (q0) edge node [left] {$a$} (q1)
         (q1) edge node [below] {$b$} (q2)
         (q2) edge node [above] {$a$} (q0)
         (q0) edge node [above] {$\lambda$} (q3)
         (q3) edge [loop above] node [above] {$b$} ()
         (q3) edge node [above] {$\lambda$} (q4)
         (q4) edge [bend left] node [left] {$a$} (q5)
         (q5) edge [bend left] node [right] {$b$} (q4)
         (q4) edge [bend left] node [right] {$b$} (q6)
         (q6) edge [bend left] node [left] {$a$} (q4)
         (q4) edge node [above] {$\lambda$} (q7)
         (q7) edge [loop above] node [above] {$a$} ();
        \end{tikzpicture}
    \end{basedtikz}
\end{itemize}

\textbf{Punto 2: }Sea $\Sigma=\{0,1\}.$ Diseñar AFN-$\lambda$ que acepten los siguientes lenguajes:

\begin{itemize}[label={$\bullet$}]
    \item $(1\cup01\cup001)^*0^*1^*0^+.$\\

    El único autómata que puede resultar un poco complicado de construir es el de la expresión inicial $(1\cup01\cup001)^*$ pero basta con 3 estados, luego simplemente es realizar una construcción en serie.
    \begin{basedtikz}
    \centering
        \begin{tikzpicture}[node distance = 2.5cm, on grid, auto]
         \node (q0) [state, initial] {$q_0$};
         \node (q1) [state, below of=q0] {$q_1$};
         \node (q2) [state, right of=q1] {$q_2$};
         \node (q3) [state, right of=q0] {$q_3$};  
         \node (q4) [state, right of=q3] {$q_4$}; 
         \node (q5) [state, right of=q4, accepting] {$q_5$};
         \path[thick]
         (q0) edge [loop above] node [above] {$1$} ()
         (q0) edge [bend right] node [left] {$0$} (q1)
         (q1) edge [bend right] node [right] {$1$} (q0)
         (q1) edge node [below] {$0$} (q2)
         (q2) edge node [above] {$1$} (q0)
         (q0) edge node [above] {$\lambda$} (q3)
         (q3) edge [loop above] node [above] {$0$} ()
         (q3) edge node [above] {$\lambda$} (q4)
         (q4) edge [loop above] node [above] {$1$} ()
         (q4) edge node [above] {$0$} (q5)
         (q5) edge [loop above] node [above] {$0$} ();
        \end{tikzpicture}
    \end{basedtikz}

    \item $0^+1(010)^*(01\cup10)^*1^+.$\\

    Nuevamente una construcción en serie basta.
     
    \begin{basedtikz}
    \centering
        \begin{tikzpicture}[node distance = 2.5cm, on grid, auto]
         \node (q0) [state, initial] {$q_0$};
         \node (q1) [state, right of=q0] {$q_1$};
         \node (q2) [state, right of=q1] {$q_2$};
         \node (q3) [state, below of=q2] {$q_3$};  
         \node (q4) [state, left of=q3] {$q_4$}; 
         \node (q5) [state, right of=q2] {$q_5$};
         \node (q6) [state, above of=q5] {$q_6$};
         \node (q7) [state, below of=q5] {$q_7$};
         \node (q8) [state, right of=q5, accepting] {$q_8$};
         \path[thick]
         (q0) edge node [above] {$0$} (q1)
         (q1) edge [loop above] node [above] {$0$} ()
         (q1) edge node [above] {$1$} (q2)
         (q2) edge node [right] {$0$} (q3)
         (q3) edge node [below] {$1$} (q4)
         (q4) edge node [above] {$0$} (q2)
         (q2) edge node [above] {$\lambda$} (q5)
         (q5) edge [bend left] node [left] {$0$} (q6)
         (q6) edge [bend left] node [right] {$1$} (q5)
         (q5) edge [bend left] node [right] {$1$} (q7)
         (q7) edge [bend left] node [left] {$0$} (q5)
         (q5) edge node [above] {$1$} (q8)
         (q8) edge [loop above] node [above] {$1$} ();
        \end{tikzpicture}
    \end{basedtikz}

    \item $(101)^*\cup1^*(1\cup10)^*0^+(01\cup10)^*.$\\

    Observe que en este caso usaremos tanto la construcción en paralelo como la construcción en serie ya que uno de los caminos es relativamente complicado de construir pero con esta construcción es mucho mas sencillo.
    \begin{basedtikz}
    \centering
        \begin{tikzpicture}[node distance = 2.5cm, on grid, auto]
         \node (q0) [state, initial] {$q_0$};
         \node (q1) [state, above right of=q0] {$q_1$};
         \node (q2) [state, right of=q1] {$q_2$};
         \node (q3) [state, right of=q2, accepting] {$q_3$};
         \node (q4) [state, below right of=q0] {$q_4$};
         \node (q5) [state, right of=q4] {$q_5$};
         \node (q6) [state, below of=q5] {$q_6$};
         \node (q7) [state, right of=q5] {$q_7$};
         \node (q8) [state, right of=q7, accepting] {$q_8$};
         \node (q9) [state, above of=q8] {$q_9$};
         \node (q10) [state, below of=q8] {$q_{10}$};
         \path[thick]
         (q0) edge [bend left] node [above] {$\lambda$} (q1)
         (q1) edge node [above] {$1$} (q2)
         (q2) edge node [above] {$0$} (q3)
         (q3) edge [bend right=45] node [above] {$1$} (q1)
         (q0) edge [bend right] node [below] {$\lambda$} (q4)
         (q4) edge [loop above] node [above] {$1$} ()
         (q4) edge node [above] {$\lambda$} (q5)
         (q5) edge [loop above] node [above] {$1$} ()
         (q5) edge [bend right] node [left] {$1$} (q6)
         (q6) edge [bend right] node [right] {$0$} (q5)
         (q5) edge node [above] {$0$} (q7)
         (q7) edge [loop above] node [above] {$0$} ()
         (q7) edge node [above] {$\lambda$} (q8)
         (q8) edge [bend left] node [left] {$0$} (q9)
         (q9) edge [bend left] node [right] {$1$} (q8)
         (q8) edge [bend left] node [right] {$1$} (q10)
         (q10) edge [bend left] node [left] {$0$} (q8);
        \end{tikzpicture}
    \end{basedtikz}
\end{itemize}

$\hfill \blacklozenge$